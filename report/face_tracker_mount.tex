\documentclass{article}
\usepackage{amsmath}
\title{Face Tracker}
\author{wandyezj}

\usepackage{comment}
\usepackage[yyyymmdd]{datetime}
\renewcommand{\dateseparator}{--}


\usepackage{verbatim}

\usepackage[margin=0.5in]{geometry}

\usepackage{enumitem}

\usepackage{listings}
\usepackage{color}

\definecolor{dkgreen}{rgb}{0,0.6,0}
\definecolor{gray}{rgb}{0.5,0.5,0.5}
\definecolor{mauve}{rgb}{0.58,0,0.82}

	\lstset{frame=tb,
	language=Java,
	aboveskip=3mm,
	belowskip=3mm,
	showstringspaces=false,
	columns=flexible,
	basicstyle={\small\ttfamily},
	numbers=none,
	numberstyle=\tiny\color{gray},
	keywordstyle=\color{blue},
	commentstyle=\color{dkgreen},
	stringstyle=\color{mauve},
	breaklines=true,
	breakatwhitespace=true,
	tabsize=3
}


\usepackage{graphicx}
%Path in Windows format:
\graphicspath{ {images/} }

\usepackage{subfig}

\usepackage{hyperref}
\hypersetup{
	colorlinks=true,
	linkcolor=blue,
	filecolor=magenta,      
	urlcolor=cyan,
}

\begin{document}
	\maketitle
	\tableofcontents
	
	
	
	\clearpage
	
	\section{Overview}
	
	Youtube Video Link: \href{https://youtu.be/De32CNrqPDQ}{https://youtu.be/De32CNrqPDQ}\newline
	Github Links: \newline
	\href{https://github.com/wandyezj/face_tracker_mount}{https://github.com/wandyezj/face\_tracker\_mount}\newline
	\href{https://github.com/wandyezj/ArduinoFaceTracker}{https://github.com/wandyezj/ArduinoFaceTracker}\newline
	\newline
	Model Overview in face\_tracker\_mount Repository:
	\newline
	\href{https://github.com/wandyezj/face\_tracker\_mount/blob/master/cad/motor\_mount.stl}{Motor Mount}\newline
	\href{https://github.com/wandyezj/face\_tracker\_mount/blob/master/cad/ultrasonic\_mount.stl}{Ultrasonic Mount}\newline
	\newline	
	Code Overview in ArduinoFaceTracker Repository:
	\newline
	ArduinoHardwareControl - RedBearDuo Arduino Hardware code
	\newline
	FaceTrackerBLE - Android Application Code
	\newline
	\newline
	
	% (i) provides an overview of your design along with key measurements 
	

	Overview:
	\newline
	The Design improves upon the original by providing a more elegant attachment mechanism for the motor to the box and the ultrasonic mount to the motor than tape that was used in the previous prototype.
	\newline
	\newline
	Key Measurements (additional details in .f3d model files next to the .stl files in the face\_tracker\_mount repository):
	\begin{itemize}
		\item Motor Length: 23 mm
		\item Motor Width: 13 mm
		\item Motor Height: 15 mm
		\item Motor Gap: 4 mm
		\item Ultrasonic Hole Diameter: 1 mm
		\item Ultrasonic Width: 2 mm
		\item Ultrasonic Height: 10 mm
		\item Wall Width Throughout: 1 mm
	\end{itemize}

	\clearpage

	\begin{figure}[h!]
		\centering
		\hfill
		\subfloat[Motor Mount Side]
		{\includegraphics[angle=270,width=0.4\textwidth]{motor_mount_side_model}\label{fig:motor_mount_side_model}}
		\hfill
		\subfloat[Motor Mount Top]
		{\includegraphics[angle=180,width=0.4\textwidth]{motor_mount_top_model}\label{fig:motor_mount_top_model}}
		\hfill
		\hfill
		\caption{Motor Mount Model}
		\label{fig:motor_mount_model}
	\end{figure}
	
	\begin{figure}[h!]
		\centering
		\hfill
		\subfloat[Ultrasonic Mount Side]
		{\includegraphics[angle=0,width=0.4\textwidth]{ultrasonic_mount_side_model}\label{fig:ultrasonic_mount_side_model}}
		\hfill
		\subfloat[Ultrasonic Mount Top]
		{\includegraphics[angle=0,width=0.4\textwidth]{ultrasonic_mount_top_model}\label{fig:ultrasonic_mount_top_model}}
		\hfill
		\hfill
		\caption{Ultrasonic Mount Model}
		\label{fig:ultrasonic_mount_model}
	\end{figure}
	
	
	
	\begin{figure}[h!]
		\centering
		\hfill
		\subfloat[Mounts Side]
		{
			\includegraphics[angle=270,width=0.4\textwidth]{mounts_side}
			\label{fig:mounts_side}
		}
		\hfill
		\subfloat[Mounts Top]
		{
			\includegraphics[angle=270,width=0.4\textwidth]{mounts_top}
			\label{fig:mounts_top}
		}
		\hfill
		\hfill
		\caption{Mounts}
		\label{fig:mounts}
	\end{figure}
	
	
	\begin{figure}[h!]
		\centering
		\hfill
		\subfloat[Mounted Front]
		{
			\includegraphics[angle=270,width=0.3\textwidth]{mounted_front}
			\label{fig:mounted_front}
		}
		\hfill
		\subfloat[Mounted Side]
		{
			\includegraphics[angle=270,width=0.3\textwidth]{mounted_side}
			\label{fig:mounted_side}
		}
		\hfill
		\subfloat[Mounted Top]
		{
			\includegraphics[angle=270,width=0.3\textwidth]{mounted_top}
			\label{fig:mounted_top}
		}
		\hfill
		\hfill
		\caption{Mounts}
		\label{fig:mounted}
	\end{figure}
	
	
	

	

	
	\clearpage
	\section{Creative Feature}
	%(ii) describes your creative feature; 
	
	The creative feature is the incorporation of smile detection and playing recordings appropriate to the context.
	A smile triggers a recording of the phrase "Turn that frown upside down".
	A lack of smile triggers a recording of the phrase "If you're happy and you know it clap your hands".
 	While this feature is entertaining it is also annoying, the Voice Activated "Off" command turns this feature off.
	
	
	\section{Challenges}
	%(iii) enumerates key struggles and challenges; 
	\begin{itemize}
		\item Learning to use Autodesk Fusion 360 was tricky since I had never used any CAD program before.
		\item Finding a space to print the models in the maker space took a while.
		\item Designing the simplest and most minimal mount possible on paper.
	\end{itemize}
	
	
	\section{Reflection}
		%(v) reflects on what you learned.
	\begin{itemize}
		\item 3D printing is pretty cool.
		\item It's important to understand the precision of the 3D printer and to leave enough room to account for it's limitations.
		\item Simple designs take much less time to 3D print and have fewer issues.
		\item It's possible to incorporate artifacts of the 3D printing as useful parts of the design.
		\item It's extremely important to have an off switch feature since repetitive sounds can get annoying.
		\item It's important to get the measurements right the first time, because 3D printing takes forever.
		\item 3D printers can print multiple parts at the same time.
	\end{itemize}
	

\end{document}